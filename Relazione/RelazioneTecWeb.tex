\documentclass{article}
\usepackage[utf8]{inputenc}
\usepackage[italian]{babel}
\usepackage{graphicx}
\usepackage{fancyhdr}
\usepackage{lastpage}
\usepackage{hyperref}
\hypersetup{
	colorlinks,
	citecolor=black,
	filecolor=black,
	linkcolor=black,
	urlcolor=blue
}

% Header e footer
\pagestyle{fancy}
\fancyhf{}
\fancyhead[L]{Sushi Nakamura}
\fancyhead[R]{\leftmark}
\fancyfoot[R]{pagina \thepage\ di \pageref{LastPage}}
\renewcommand{\headrulewidth}{1pt}
\renewcommand{\footrulewidth}{1pt}

\begin{document}
	% Frontespizio
	\begin{titlepage}
		\begin{figure}[http]
			\centering
			\includegraphics[width=6cm]{logo.jpg}
		\end{figure}
	
		\vspace*{2cm}
		
		{\huge\bfseries\centerline{Sushi Nakamura} }
		\centerline{Progetto di Tecnologie Web A.A. 2019/2020}
		
		\vspace*{1cm}
		{\bfseries \centerline{Informazioni sul gruppo}}
		\begin{center}
			\begin{tabular}{ c|l } 
				Membri & Dindinelli Alessandro - 1170457\\ 
				& Frison Nicolò - 1147682\\ 
				& Giardina Mirco - 1136663\\
				& Tommasin Alessandro - 1189293\\ 
			\end{tabular}
		\end{center}
		
		\vspace*{\fill}
		
	\end{titlepage}
	
	% Pagina indici
	\clearpage
	\renewcommand*\contentsname{Indice}
	\tableofcontents	
	\newpage
	
	% Contenuto
	\section{Introduzione}
		\subsection{Abstract}
			Il sito web \textbf{Sushi Nakamura} è stato sviluppato per permettere all'omonimo ristorante di Padova un mezzo per promuovere sè stesso ed il suo nuovo servizio di take away.
			Nel sito è possibile reperire tutte le informazioni riguardanti i contatti ed i prodotti che possono essere acquistati.
			Inoltre l'amministratore ha la possibilità di inserire, rimuovere e modificare eventuali articoli in vendita e news che possono essere visualizzate dagli utenti.
	\section{Analisi}
		\subsection{Analisi dell'utenza}
		\subsection{Casi d'uso}
			I casi d'uso possono essere riassunti sotto le seguenti categorie:
			\subsubsection{Utente non autenticato}
			\subsubsection{Utente generico autenticato}
			\subsubsection{Amministratore}
	\section{Progettazione}
		\subsection{Obiettivi}
		Gli obiettivi principali perseguiti durante la progettazione del sito sono i seguenti:
		\begin{itemize}
		    \item \textbf{Separazione tra struttura, presentazione e comportamento:}
		    Obiettivo fondamentale, in quanto raggiungerlo permette di soddisfare più agevolmente anche gli altri punti. 
		    La struttura è stata realizzata con documenti in XHTML 1.0 Strict dove possibile, così da garantire una maggiore retrocompatibilità con vecchi browsers, ed HTML5 dove si sono ritenute necessarie le funzionalità aggiuntive permesse dal linguaggio. 
		    La presentazione è stata sviluppata con fogli di stile CSS linkati, mentre il comportamento con script esterni realizzati in Javascript e PHP. In questo modo la struttura non dovrà cambiare, anche a seguito di modifiche alla presentazione del sito.
		    Tutto il codice redatto è stato scritto secondo le raccomandazioni W3C, accertando poi che siano state rispettate, validando HTML e CSS con i rispettivi tool di W3C.
		    \item \textbf{Accessibilità:}
		    Il sito deve poter essere fruibile agevolmente dal maggior numero di utenti possibile, compresi quelli con differenti tipi di disabilità. Per garantire una buona accessibilità, alcune misure adottate sono:
		    \begin{itemize}
		        \item Uso dei tabindex nel menù di navigazione;
		        \item Testo alternativo per le immagini;
		        \item Assenza di link circolari;
		        \item Testi e link con buoni livelli di contrasto;
		        \item Uso dell'attributo lang per testi non in italiano;
		    \end{itemize}
		    \item \textbf{Fluidità:}
		    Il sito deve poter essere consultabile tramite varie tipologie di dispositivi, tra cui PC desktop, tablet e smartphone. Bisogna quindi garantire una buona adattabilità alle differenti dimensioni di schermo.
		    \item \textbf{Fruibilità:}
		    Per realizzare un sito navigabile intuitivamente si sono seguite alcune linee guida comuni nel web, come ad esempio:
		    \begin{itemize}
		        \item Sfruttare un layout ben strutturato, che faciliti l'individuazione del contenuto di interesse;
		        \item Agevolare lo scroll tramite link relativi per raggiungere diversi punti di una stessa pagina;
		        \item Mantenere colorazioni diverse per link visitati e non;
		    \end{itemize}
		\end{itemize}
		\subsection{Layout}
		\subsection{Accessibilità}
	\section{Implementazione}
		\subsection{Linguaggi}
			\subsubsection{XHTML 1.0 Strict e HTML5}
			\subsubsection{CSS}
			\subsubsection{PHP}
			\subsubsection{SQL}
			Sql è stato usato per codificare il database. Si rimanda al file \textit{creazione\_database.sql} nella cartella \textit{Database} [\href{https://github.com/Mirco469/ProgettoSushi/tree/master/Database}{url}] della repository per il file di costruzione del database. Di seguito il diagramma ER del database:\newline
			\includegraphics[width=12cm]{DiagrammaER.png}
			\subsubsection{JavaScript}
	\section{Fase di test}
		\subsection{Strumenti usati}
			\subsubsection{W3C HTML Validator}
				Le pagine html sono state validate usando il validatore fornito dall'organizazione W3C per garantire la corretta visualizzazione del contenuto della pagina senza fare entrare i browser in {\bfseries Quirks Mode}. \'E stato usato anche per validare il risultato delle pagine php incollando il risultato ottenuto facendo eseguire lo script php.
			\subsubsection{W3C CSS Validator}
			\subsubsection{TotalValidator}
			\subsubsection{SonarCloud}
				Servizio integrato con GitHub per la verifica del codice nella repository. Ad ogni push veniva fatta un {\bfseries analisi statica} del codice alla ricerca di problemi e vulnerabilità come ad esempio un problema comune è stata la ripetizione rindondate di regole css. Questa fase di test era bloccante, ovvero perchè il codice venisse aggiunto dovevano prima essere risolti i problemi.
	\section{Organizzazione del lavoro}
		Il progetto è stato suddiviso in modo tale che ogni membro avesse la possibilità di creare sia alcune pagine HTML che il relativo CSS, facendo da verificatore nelle pagine degli altri membri.
		Lo stesso può essere detto per quanto concerne PHP, JavaScript e la creazione ed il popolamento del database.
		La sviluppo può essere seguito nella repossitory di GitHub utilizzata:
		\newline
		\newline
		\centerline{ \url{https://github.com/Mirco469/ProgettoSushi}}

\end{document}
