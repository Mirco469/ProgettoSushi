\documentclass{article}
\usepackage[utf8]{inputenc}
\usepackage[italian]{babel}
\usepackage{graphicx}
\usepackage{fancyhdr}
\usepackage{lastpage}
\usepackage{hyperref}
\hypersetup{
	colorlinks,
	citecolor=black,
	filecolor=black,
	linkcolor=black,
	urlcolor=blue
}

% Header e footer
\pagestyle{fancy}
\fancyhf{}
\fancyhead[L]{Sushi Nakamura}
\fancyhead[R]{\leftmark}
\fancyfoot[R]{pagina \thepage\ di \pageref{LastPage}}
\renewcommand{\headrulewidth}{1pt}
\renewcommand{\footrulewidth}{1pt}

\begin{document}
	% Frontespizio
	\begin{titlepage}
		\begin{figure}[http]
			\centering
			\includegraphics[width=6cm]{logo.jpg}
		\end{figure}
	
		\vspace*{2cm}
		
		{\huge\bfseries\centerline{Sushi Nakamura} }
		\centerline{Progetto di Tecnologie Web A.A. 2019/2020}
		
		\vspace*{1cm}
		{\bfseries \centerline{Informazioni sul gruppo}}
		\begin{center}
			\begin{tabular}{ c|l } 
				Membri & Dindinelli Alessandro - XXXXXXX\\ 
				& Frison Nicolò - 1147682\\ 
				& Giardina Mirco - 1136663\\
				& Tommasin Alessandro - 1189293\\ 
			\end{tabular}
		\end{center}
		
		\vspace*{\fill}
		
	\end{titlepage}
	
	% Pagina indici
	\clearpage
	\renewcommand*\contentsname{Indice}
	\tableofcontents	
	\newpage
	
	% Contenuto
	\section{Introduzione}
		\subsection{Abstract}

	
	\section{Analisi}
		\subsection{Analisi dell'utenza}
		\subsection{Casi d'uso}
			I casi d'uso possono essere riassunti sotto le seguenti categorie:
			\subsubsection{Utente non autenticato}
			\subsubsection{Utente generico autenticato}
			\subsubsection{Amministratore}
	\section{Progettazione}
		\subsection{Obiettivi}
		\subsection{Layout}
		\subsection{Accessibilità}
	\section{Implementazione}
		\subsection{Linguaggi}
			\subsubsection{XHTML 1.0 Strict e HTML5}
			\subsubsection{CSS}
			\subsubsection{PHP}
			\subsubsection{SQL}
			\subsubsection{JavaScript}
	\section{Fase di test}
		\subsection{Strumenti usati}
			\subsubsection{W3C HTML Validator}
				Le pagine html sono state validate usando il validatore fornito dall'organizazione W3C per garantire la corretta visualizzazione del contenuto della pagina senza fare entrare i browser in {\bfseries Quirks Mode}. \'E stato usato anche per validare il risultato delle pagine php incollando il risultato ottenuto facendo eseguire lo script php.
			\subsubsection{W3C CSS Validator}
			\subsubsection{TotalValidator}
			\subsubsection{SonarCloud}
				Servizio integrato con GitHub per la verifica del codice nella repository. Ad ogni push veniva fatta un {\bfseries analisi statica} del codice alla ricerca di problemi e vulnerabilità come ad esempio un problema comune è stata la ripetizione rindondate di regole css. Questa fase di test era bloccante, ovvero perchè il codice venisse aggiunto dovevano prima essere risolti i problemi.
	\section{Organizzazione del lavoro}
		

\end{document}